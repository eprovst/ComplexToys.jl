\documentclass[a4paper]{article}

\usepackage{amsmath, amssymb}
\usepackage{lmodern}
\usepackage{geometry}
\usepackage{parskip}
\usepackage{microtype}
\usepackage{xcolor}
\usepackage{float}
\usepackage[hyphens]{url}
\usepackage[hidelinks,unicode]{hyperref}
\usepackage{graphicx, subcaption}
\usepackage[
	usefamily={jl,julia},
	gobble=auto,
	pygopt={style=xcode}]{pythontex}
\usepackage[british]{babel}

\title{\vspace*{-1em}The \texttt{CATools.jl} guide}
\author{Evert Provoost\\\normalsize evert.provoost@kuleuven.be}
\date{}

\begin{document}
\maketitle

\section{Installation}

\texttt{CATools.jl} (short for complex analysis tools) is a collection of
plotting tools to aid the study of complex analysis. It is implemented in Julia,
a modern and---at least in the author's opinion---a rather pleasant language to
use.  No prior knowledge of Julia is needed to go through this document as
concepts will be introduced where needed. If Julia is not yet installed on your
system, you can use the usual method of your operating system or download it
from \url{https://julialang.org}. We strongly recommend you use at least version
1.9 or later!

To install the plotting routines, you enter the following commands in the Julia
REPL (read eval print loop).  \textbf{Note:} Depending on the speed of your
machine, this might take a while. Grab a cup of tea or coffee and take a minute
to enjoy the outdoors. :-)

\begin{juliaverbatim}
	using Pkg
	pkg"add https://gitlab.kuleuven.be/u0158737/catools.git"
\end{juliaverbatim}

\section{A first phase plot}

To start producing plots, you first have to load the package you just installed
with the \jlv|using| keyword.

\begin{juliaverbatim}
	using CATools
\end{juliaverbatim}

In the remainder of this tutorial we will discuss several plots, starting with
the phase plot (we will explain the name \jlv|domaincolor| in the next
section). On such a plot, we display the phase, or argument, by painting
$\arg z = 0$ red, $\arg z = \frac{\pi}{2}$ lime green, $\arg z = \pi$ cyan,
$\arg z = \frac{3\pi}{2}$ purple, etc.

\begin{juliaverbatim}
	domaincolor(z -> z)
\end{juliaverbatim}
\begin{figure}[H]
	\centering
	\includegraphics[width=.45\textwidth]{figures/phase-id.png}
\end{figure}

We note that a zero, going anticlockwise, gives red, green and finally blue. A
pole on the other hand (below one of second order) is surrounded by red, blue
and then green.

\begin{juliaverbatim}
	domaincolor(z -> 1/z^2)
\end{juliaverbatim}
\begin{figure}[H]
	\centering
	\includegraphics[width=.45\textwidth]{figures/phase-pole.png}
\end{figure}

Note the first bit of Julia syntax you will need: an anonymous function (a
function you can easily pass as an argument) is written as \jlv|z -> f(z)|,
analogous to $z \mapsto f(z)$ in mathematics.

It is perhaps interesting to know that this reversal of the phase relates to the
complex conjugate. The function $z \mapsto \frac{1}{z}$ is namely the
composition of $z \mapsto \frac{1}{\bar{z}} = \frac{1}{|z|} e^{i\arg z}$ and
$z \mapsto \bar{z} = |z| e^{-i\arg z}$. In the first step, the unit circle is
turned inside out; the second step mirrors the image along the real axis.

This is easily seen by looking at the effect these steps have on a square, or
algebraically by noting that only the final step modifies the phase.

\begin{juliaverbatim}
	domaincolor(z -> z, 2, box=(.4,.6+.2im,:white))
	domaincolor(z -> 1/conj(z), 2, box=(.4,.6+.2im,:white))
	domaincolor(z -> 1/z, 2, box=(.4,.6+.2im,:white))
\end{juliaverbatim}
\begin{figure}[H]
	\centering
	\begin{subfigure}{.3\textwidth}
		\centering
		\includegraphics[width=\textwidth]{figures/box-1.png}
		$z \mapsto z$
	\end{subfigure}
	\hspace{1ex}
	\begin{subfigure}{.3\textwidth}
		\centering
		\includegraphics[width=\textwidth]{figures/box-2.png}
		$z \mapsto \frac{1}{\bar{z}}$
	\end{subfigure}
	\hspace{1ex}
	\begin{subfigure}{.3\textwidth}
		\centering
		\includegraphics[width=\textwidth]{figures/box-3.png}
		$z \mapsto \frac{1}{z}$
	\end{subfigure}
\end{figure}

In the above code, the second argument fixes the range of the axes. For the
specific details about the \jlv|box| keyword argument we refer to the
documentation (try: \jlv|? domaincolor|, more on this later). Also note the use
of \jlv|im| as imaginary unit in Julia.

\section{Domain colouring and modular surfaces}

Given that $z \mapsto \frac{1}{z}$ and $z \mapsto \bar{z}$ have identical phase,
a phase plot alone is not always sufficient to study a function's behaviour.
For this reason it is common to add contour lines of the magnitude (usually more
specifically of its logarithm). The resulting figure is called a domain
colouring (hence \jlv|domaincolor|). We can add the contours using the option
\jlv|abs=true|. This way we can distinguish the pole (left, with
\emph{increasing} lightness between contours) and zero (right, with
\emph{decreasing} lightness between contours) in the below example.

\begin{juliaverbatim}
	domaincolor(z -> (conj(z) - .5)/(z + .5), abs=true)
\end{juliaverbatim}
\begin{figure}[H]
	\centering
	\includegraphics[width=.45\textwidth]{figures/candp.png}
\end{figure}

Of course there are many variations on this idea. One could, for example, paint
zero black and infinity white. This is for instance useful to illustrate the
Casorati--Weierstrass theorem.

\begin{juliaverbatim}
	domaincolor(z -> exp(1/z), .2, abs=Inf)
\end{juliaverbatim}
\begin{figure}[H]
	\centering
	\includegraphics[width=.45\textwidth]{figures/ess.png}
\end{figure}

Aside from what is illustrate here, many other options are available, for
instance grid lines and colour vision deficiency friendly phase plots. For a
list of all the available options you can consult a function's documentation
using \jlv|?|, for example:

\begin{juliaverbatim}
	? domaincolor
\end{juliaverbatim}

The 2D plots included in \texttt{CATools.jl} come from a different Julia package
called \texttt{DomainColoring.jl}; visit
\url{https://eprovst.github.io/DomainColoring.jl} for more information and
examples.

Yet another way to represent magnitude and phase at the same time, is by moving
the former to the third dimension, which results in a (painted) modular surface.
Another essential singularity we can visualize is
$z \mapsto \sin\left(\frac{1}{z}\right)$ near $0$.

\begin{juliaverbatim}
	modularsurface(z -> sin(1/z))
\end{juliaverbatim}
\begin{figure}[H]
	\centering
	\includegraphics[width=.45\textwidth]{figures/sin-ess.png}
\end{figure}

For this example---as with all other 3D plots---it is worthwhile to plot the
figure yourself. This additionally allows you to rotate the plot.

\section{Branch points and Riemann surfaces}

When looking at the domain colouring of $z \mapsto z^2$ we see the same values
appear twice in the complex plane.

\begin{juliaverbatim}
	domaincolor(z -> z^2, abs=true)
\end{juliaverbatim}
\begin{figure}[H]
	\centering
	\includegraphics[width=.45\textwidth]{figures/dc-pow2.png}
\end{figure}

The inverse map $z \mapsto z^{\frac{1}{2}}$ will hence take two values in any
given point. An exception is $0$, where the inverse map is unique. Such a point
$z_0$ where a map has $n$ values (here $n=1$), but every neighbour has strictly
more than $n$ values, is called a \emph{branch point} (more precisely: every
neighbourhood of $z_0$ has at least one point where the map takes at least $n+1$
values).

If we want to arrive at a single valued function, we will have to choose between
one of the two values. The usual approach for a continuous function is to pick
an arbitrary curve (the \emph{branch cut}) connecting two branch points, and to
require continuity everywhere except when crossing this curve. This results in a
set of single valued functions (so called \emph{branches}) which can be attached
along this curve in a continuous fashion.

For $z \mapsto z^\frac{1}{2}$, the usual choice of branch cut is the negative
real axis (connecting the branch points $0$ and $\infty$), as is also done in
Julia. This is a result of the convention to use
${|z|}^{\frac{1}{n}} e^{i \frac{\mathrm{Arg}\,z}{n}}$ as the so-called
\emph{principal value} of the $n$th root, with $\mathrm{Arg}\,z \in (-\pi, \pi]$
the principal value of the argument. Analogously, the principal value of the
logarithm is given by $\mathrm{Log}\,z = \log |z| + i\,\mathrm{Arg}\, z$, which
is of course compatible with the preceding.

\begin{juliaverbatim}
	domaincolor(z -> z^(1/2), abs=true)
\end{juliaverbatim}
\begin{figure}[H]
	\centering
	\includegraphics[width=.45\textwidth]{figures/bc-sqrt.png}
\end{figure}

Additionally, we could also try to make a plot of all the values at the same
time. Note that this is a four dimensional object. For $w = f(z)$ we namely have
$(\mathrm{Re}\,z, \mathrm{Im}\,z, \mathrm{Re}\,w, \mathrm{Im}\,w)$ as
graph. (One can, of course, also use the phases and magnitudes.)

When we project this orthogonally on the first three components and colour
the resulting surface according to the removed component, we get the
following for $z \mapsto z^{\frac{1}{2}}$.

\begin{juliaverbatim}
	riemannpow(1//2)
\end{juliaverbatim}
\begin{figure}[H]
	\centering
	\includegraphics[width=.45\textwidth]{figures/rm-sqrt.png}
\end{figure}

Here \jlv|1//2| is Julia notation for the exact rational number
$\frac{1}{2}$. On the negative real axis the graph seems to intersect itself,
but this is merely an artefact of the chosen projection. The colour is in fact
distinct. Note that indeed each point, except for the branch point $0$, has two
distinct values.

We encourage you to play around with different rational numbers and to look what
does (and what does not) change the number of values. Also try an irrational
power; note that it never reattaches to itself, \emph{it takes an infinite
	number of values!}

A `simpler' example of a function that takes infinitely many values is
$\log z = \log |z| + i\arg z$.  Using a slightly different projection, we
get a rather nice looking staircase.

\begin{juliaverbatim}
	riemannlog()
\end{juliaverbatim}
\begin{figure}[H]
	\centering
	\includegraphics[width=.45\textwidth]{figures/rm-log.png}
\end{figure}

\section{On the Riemann sphere}

Finally, \texttt{CATools.jl} is also able to plot a function on the Riemann
sphere.  Let us, to conclude, again look at the sine, with a zero at the origin
(south pole) and an essential singularity at infinity (north pole).

\begin{juliaverbatim}
	riemannsphere(sin)
\end{juliaverbatim}
\begin{figure}[H]
	\centering
	\includegraphics[width=0.75\textwidth]{figures/rms-sin.png}
\end{figure}

\end{document}
